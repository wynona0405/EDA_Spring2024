% Options for packages loaded elsewhere
\PassOptionsToPackage{unicode}{hyperref}
\PassOptionsToPackage{hyphens}{url}
%
\documentclass[
  ignorenonframetext,
]{beamer}
\usepackage{pgfpages}
\setbeamertemplate{caption}[numbered]
\setbeamertemplate{caption label separator}{: }
\setbeamercolor{caption name}{fg=normal text.fg}
\beamertemplatenavigationsymbolsempty
% Prevent slide breaks in the middle of a paragraph
\widowpenalties 1 10000
\raggedbottom
\setbeamertemplate{part page}{
  \centering
  \begin{beamercolorbox}[sep=16pt,center]{part title}
    \usebeamerfont{part title}\insertpart\par
  \end{beamercolorbox}
}
\setbeamertemplate{section page}{
  \centering
  \begin{beamercolorbox}[sep=12pt,center]{part title}
    \usebeamerfont{section title}\insertsection\par
  \end{beamercolorbox}
}
\setbeamertemplate{subsection page}{
  \centering
  \begin{beamercolorbox}[sep=8pt,center]{part title}
    \usebeamerfont{subsection title}\insertsubsection\par
  \end{beamercolorbox}
}
\AtBeginPart{
  \frame{\partpage}
}
\AtBeginSection{
  \ifbibliography
  \else
    \frame{\sectionpage}
  \fi
}
\AtBeginSubsection{
  \frame{\subsectionpage}
}
\usepackage{amsmath,amssymb}
\usepackage{iftex}
\ifPDFTeX
  \usepackage[T1]{fontenc}
  \usepackage[utf8]{inputenc}
  \usepackage{textcomp} % provide euro and other symbols
\else % if luatex or xetex
  \usepackage{unicode-math} % this also loads fontspec
  \defaultfontfeatures{Scale=MatchLowercase}
  \defaultfontfeatures[\rmfamily]{Ligatures=TeX,Scale=1}
\fi
\usepackage{lmodern}
\ifPDFTeX\else
  % xetex/luatex font selection
\fi
% Use upquote if available, for straight quotes in verbatim environments
\IfFileExists{upquote.sty}{\usepackage{upquote}}{}
\IfFileExists{microtype.sty}{% use microtype if available
  \usepackage[]{microtype}
  \UseMicrotypeSet[protrusion]{basicmath} % disable protrusion for tt fonts
}{}
\makeatletter
\@ifundefined{KOMAClassName}{% if non-KOMA class
  \IfFileExists{parskip.sty}{%
    \usepackage{parskip}
  }{% else
    \setlength{\parindent}{0pt}
    \setlength{\parskip}{6pt plus 2pt minus 1pt}}
}{% if KOMA class
  \KOMAoptions{parskip=half}}
\makeatother
\usepackage{xcolor}
\newif\ifbibliography
\usepackage{graphicx}
\makeatletter
\def\maxwidth{\ifdim\Gin@nat@width>\linewidth\linewidth\else\Gin@nat@width\fi}
\def\maxheight{\ifdim\Gin@nat@height>\textheight\textheight\else\Gin@nat@height\fi}
\makeatother
% Scale images if necessary, so that they will not overflow the page
% margins by default, and it is still possible to overwrite the defaults
% using explicit options in \includegraphics[width, height, ...]{}
\setkeys{Gin}{width=\maxwidth,height=\maxheight,keepaspectratio}
% Set default figure placement to htbp
\makeatletter
\def\fps@figure{htbp}
\makeatother
\setlength{\emergencystretch}{3em} % prevent overfull lines
\providecommand{\tightlist}{%
  \setlength{\itemsep}{0pt}\setlength{\parskip}{0pt}}
\setcounter{secnumdepth}{-\maxdimen} % remove section numbering
\ifLuaTeX
  \usepackage{selnolig}  % disable illegal ligatures
\fi
\IfFileExists{bookmark.sty}{\usepackage{bookmark}}{\usepackage{hyperref}}
\IfFileExists{xurl.sty}{\usepackage{xurl}}{} % add URL line breaks if available
\urlstyle{same}
\hypersetup{
  pdftitle={Crafting Reports},
  pdfauthor={John Fay \& Luana Lima},
  hidelinks,
  pdfcreator={LaTeX via pandoc}}

\title{Crafting Reports}
\subtitle{ENV 859 - Environmental Data Exploration}
\author{John Fay \& Luana Lima}
\date{Spring 2024}

\begin{document}
\frame{\titlepage}

\begin{frame}{Learning Outcomes}
\protect\hypertarget{learning-outcomes}{}
\begin{itemize}
\tightlist
\item
  Explain why we use RMarkdown \& knit to PDF
\item
  Navigate YAML \& use it to enhance to your documents
\item
  Control how \emph{code} appears in knit output: code chunk settings
\item
  Control how \emph{text} appears in knit output:markdown techniques
\item
  Understand the knitting process and how it converts Rmd to PDF, HTML,
  etc.
\end{itemize}
\end{frame}

\begin{frame}{Use of RStudio \& RMarkdown so far\ldots{}}
\protect\hypertarget{use-of-rstudio-rmarkdown-so-far}{}
\begin{itemize}
\tightlist
\item
  We have provided you with Rmd files
\item
  Write/edit code (code chunks)
\item
  Document that code (text blocks)
\item
  Generate PDFs of code and its outputs
\item
  Integrate with Git/GitHub for version control
\end{itemize}
\end{frame}

\begin{frame}{Advantanges of RMarkdown}
\protect\hypertarget{advantanges-of-rmarkdown}{}
\begin{block}{Why do we use RMarkdown?}
\protect\hypertarget{why-do-we-use-rmarkdown}{}
\begin{itemize}
\tightlist
\item
  Written in basic text (fully reproducible)
\item
  Works well with version control
\item
  Allows for code and documentation in one file
\item
  When knit, can produce tidy output in many file formats
\end{itemize}
\end{block}
\end{frame}

\begin{frame}{RMarkdown: A deeper look}
\protect\hypertarget{rmarkdown-a-deeper-look}{}
\begin{block}{The three components of RMarkdown}
\protect\hypertarget{the-three-components-of-rmarkdown}{}
\begin{itemize}
\tightlist
\item
  YAML
\item
  Text (markdown)
\item
  Code (R or ???)
\end{itemize}
\end{block}
\end{frame}

\begin{frame}[fragile]{YAML: What is it?}
\protect\hypertarget{yaml-what-is-it}{}
\begin{itemize}
\tightlist
\item
  ``Yet another markdown language''\\
  ``YAML ain't markdown language''
\item
  Always at the beginning of an Rmd
\item
  Enclosed by three dashes \texttt{-\/-\/-}
\item
  Includes instructions applied when knitting
\item
  Instructions take the form of \texttt{key}:\texttt{value} pairs
\end{itemize}
\end{frame}

\begin{frame}[fragile]{YAML, An Exercise}
\protect\hypertarget{yaml-an-exercise}{}
\begin{enumerate}
\tightlist
\item
  Create a new Rmd in R
\item
  Set the title, author, date
\item
  Set output to be PDF
\item
  View the YAML in the Rmd produced
\item
  Knit the document \& view the PDF
\item
  Add new entries:

  \begin{itemize}
  \tightlist
  \item
    \texttt{subtitle:\ "A\ draft\ copy"}
  \item
    \texttt{toc:\ true}
  \end{itemize}
\item
  Add a few markdown headers (\texttt{\#\#})
\item
  Knit and view\ldots{}
\end{enumerate}
\end{frame}

\begin{frame}{YAML, A Recap}
\protect\hypertarget{yaml-a-recap}{}
\begin{itemize}
\tightlist
\item
  YAML setting apply to the entire document
\item
  YAML is read in the knit process
\item
  YAML has even more advanced settings\\
  \href{https://raw.githubusercontent.com/rstudio/cheatsheets/main/rmarkdown.pdf}{Markdown
  Cheat sheet}\\
  \url{https://bookdown.org/yihui/rmarkdown/pdf-document.html}
\end{itemize}
\end{frame}

\begin{frame}[fragile]{The knitting process}
\protect\hypertarget{the-knitting-process}{}
\begin{itemize}
\tightlist
\item
  The knitting sequence
  \href{https://env872.github.io/assets/images/rmarkdownflow.png}{\includegraphics{img/rmarkdownflow.png}}
\item
  \texttt{Rmd}→{[}\emph{knitr}{]}→\texttt{md}:\\
  Guided by instructions in YAML \& Code chunks
\item
  \texttt{md}→{[}\emph{Pandoc}{]}→\texttt{PDF}:\\
  Markdown translated into PDF\textbar HTML\textbar Word\textbar\ldots{}
  by Pandoc
\item
  Example\ldots{}
\end{itemize}
\end{frame}

\begin{frame}[fragile]{Code Chunks: Recap}
\protect\hypertarget{code-chunks-recap}{}
\begin{itemize}
\tightlist
\item
  Where code is executed
\item
  Surrounded by triplets of tics:
  \texttt{\textasciigrave{}\textasciigrave{}\textasciigrave{}}
\item
  Chunks have attributes, set within \texttt{\{\ \ \}}\\
  For example: \texttt{\{r\ plot\ AQI,\ echo=FALSE\}}
\end{itemize}
\end{frame}

\begin{frame}[fragile]{Code Chunks: Example}
\protect\hypertarget{code-chunks-example}{}
\begin{itemize}
\tightlist
\item
  Shortcuts to create
\item
  Coding languages
\item
  Setting chunk attributes

  \begin{itemize}
  \tightlist
  \item
    \url{https://yihui.org/knitr/options/\#chunk-options}
  \item
    \url{https://bookdown.org/yihui/rmarkdown/r-code.html}
  \end{itemize}
\item
  Global code chunk attributes\\
  \texttt{knitr::opts\_chunk\$set(echo\ =\ TRUE)}
\end{itemize}
\end{frame}

\begin{frame}[fragile]{Code Chunks: Plots and Figures}
\protect\hypertarget{code-chunks-plots-and-figures}{}
\begin{itemize}
\tightlist
\item
  \texttt{fig.cap}: Set figure Caption
\item
  \texttt{fig.width}, \texttt{fig.height}: Set figure size, or\ldots{}
\item
  \texttt{fig.dim} to set both (e.g.~\texttt{fig.dim\ =\ c(5,3)})
\item
  \texttt{out.width}, \texttt{out.height}: scale figs in output docs\\
  e.g.~\texttt{out.width=80\%} to set at 80\% of page width
\item
  \texttt{fig.align}: set alignment (`left', `center', etc.)
\end{itemize}
\end{frame}

\begin{frame}[fragile]{Markdown: Recap}
\protect\hypertarget{markdown-recap}{}
\begin{block}{Common stylings:}
\protect\hypertarget{common-stylings}{}
\begin{itemize}
\tightlist
\item
  \texttt{\#}: headings
\item
  \texttt{\textgreater{}}: block quotes
\item
  Ordered and unordered lists\ldots{}
\item
  Hyperlinks\ldots{}
\end{itemize}
\end{block}
\end{frame}

\begin{frame}{Markdown: Documentation}
\protect\hypertarget{markdown-documentation}{}
\begin{itemize}
\tightlist
\item
  Quick reference
\item
  Cheatsheet
\item
  Reference Guide
\end{itemize}
\end{frame}

\begin{frame}{Markdown: Exercises}
\protect\hypertarget{markdown-exercises}{}
\begin{itemize}
\tightlist
\item
  Creating tables
\item
  Inserting images
\item
  Additional text formatting
\end{itemize}
\end{frame}

\begin{frame}[fragile]{Markdown: Useful techniques}
\protect\hypertarget{markdown-useful-techniques}{}
\begin{itemize}
\tightlist
\item
  \texttt{\textbackslash{}newpage}
\item
  \texttt{\textbackslash{}listoftables}
\item
  \texttt{\textbackslash{}listoffigurres}
\item
  \texttt{-\/-\/-}
\item
  Using the \emph{Visual Editor}!
\end{itemize}
\end{frame}

\begin{frame}[fragile]{Pretty tables with \texttt{kable}}
\protect\hypertarget{pretty-tables-with-kable}{}
\begin{itemize}
\tightlist
\item
  Part of the \texttt{knitr} package
\item
  Convert a dataframe to a markdown table\\
  Makes for aethetic output
\item
  Example\ldots{}
\end{itemize}
\end{frame}

\begin{frame}[fragile]{Exercises!}
\protect\hypertarget{exercises}{}
\begin{itemize}
\tightlist
\item
  Open \texttt{06Exercise\_Markdown.Rmd} and knit as is
\item
  Add a table of contents
\item
  Set the first code chunk to run, but:

  \begin{itemize}
  \tightlist
  \item
    code is not shown in knit output
  \item
    warnings are not shown in output
  \end{itemize}
\item
  Show summary dataframes as tables (using \texttt{kable})
\item
  Add a list of tables using \texttt{\textbackslash{}listoftables}
\end{itemize}
\end{frame}

\end{document}
